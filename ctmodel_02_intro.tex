% !TEX root = ctmodel.tex

\section{Introduction}

The goal of this paper is to articulate a realistic scenario for engineering design and iteration, and to pose a solution using a (hypothetical) modeling tool based on category theory. Criteria: (1) realistic, (2) has a complicated modeling component, ideally involving both a model of the product, the manufacturing process, and a model of the technology that maps a process to a product, (3) has a data integration component where actual data (e.g. from devices) needs to be merged.

Some possible scenarios: 
\begin{enumerate}
\item we design our own product, a modeling tool
\item traditional manufacturing such as battery production, combining schematics, database schemas, and process diagrams
\item metrics embedded in Excel, MATLAB, or SAS
\item software engineering
\end{enumerate}

To get this project started, we need:
\begin{enumerate}
\item a checklist
\item some sources, ideally at least one engineer we can ping whenever to ask for advice and details
\item software engineering
\end{enumerate}

Resources: Sub's nDim paper at \url{http://www.ndim.edrc.cmu.edu/ndim/papers/overview.pdf}, and The Synthesis of Variety at \url{https://pure.tue.nl/ws/files/1564910/466456.pdf}.

\section{Wishlist}
I'd also like to understand what a ``distributed'' approach would look like.